\section{Discretization}%
\label{sec:discretization}
Consider a spatial discretization of the incompressible Euler equations~\eqref{eq:euler_nobc_split}.
\begin{subequations}%
\label{eq:discrete_euler_nobc}
  \begin{align}
    \wn_t + \frac{1}{2} \wn \cdotn \nablan \wn + \frac{1}{2} \divn (\wn \wn^\dagger) &= -\nablan \pn \label{eq:discrete_momentum} \\
    \divn \wn &= \zeron \,, \label{eq:discrete_divergence}
  \end{align}
\end{subequations}
where $\wn$ and $\pn$ are time dependent grid functions. The form~\eqref{eq:discrete_euler_nobc} is particularly convenient for discrete energy analysis since the continuous energy analysis can be replicated. Indeed, both Lemma~\ref{lemma:convection2} and Proposition~\ref{prop:energy2} hold discretely (the proofs are identical, simply replace the normal letters by bold letters), resulting in the discrete energy:

\begin{proposition}\label{prop:discrete_energy}
  The semidiscrete incompressible Euler equations~\eqref{eq:discrete_euler_nobc} imply that
  \begin{equation}\label{eq:discrete_boundary_terms}
    \frac{d}{dt} \|\wn\|_{\Omega_h}^2
      = -\ipbnfull{\wnn}{|\wn|^2} - 2\ipbnfull{\wnn}{\pn} + 2\ipn{\pn}{\divn \wn}\,.
  \end{equation}
\end{proposition}
\begin{remark}
  We do not omit the term $2\ipn{\pn}{\divn \wn}$ in Proposition~\ref{prop:discrete_energy} even though it vanishes by~\eqref{eq:discrete_divergence}. This is because it is sometimes useful to make small perturbations to the discrete divergence (i.e.\ allow for local nonzero divergence at a boundary) in order to cancel the pressure term $-2\ipbn{\wnn}{\pn}$ (see the discrete boundary treatment in Section~\ref{sec:discrete_boundary_conditions}).
\end{remark}

For implementation purposes it is better to write~\eqref{eq:discrete_euler_nobc} in more explicit form as
  \begin{equation}
    \label{eq:discrete_euler_explicit}
    \begin{bmatrix}
      \un_t \\
      \vn_t \\
      \zeron
    \end{bmatrix}
    + \euln(\un, \vn, \pn) = \zeron \,,
  \end{equation}
  where
  \begin{equation*}
    \euln(\un, \vn, \pn) =
    \begin{bmatrix}
      \euln_1 & \euln_2 & \euln_3
    \end{bmatrix}^\dagger
  \end{equation*}
  and
  \begin{equation*}
    \begin{aligned}
      \euln_1 &= \frac{1}{2} \un \Dx \un + \frac{1}{2} \Dx \un \un + \frac{1}{2} \vn \Dy \un + \frac{1}{2} \Dy \un \vn + \Dx \pn \\
      \euln_2 &= \frac{1}{2} \vn \Dy \vn + \frac{1}{2} \Dy \vn \vn + \frac{1}{2} \un \Dx \vn + \frac{1}{2} \Dx \un \vn + \Dy \pn \\
      \euln_3 &= \Dx \un + \Dy \vn \,.
    \end{aligned}
  \end{equation*}
\noindent The discretization~\eqref{eq:discrete_euler_explicit} defines a set of differential algebraic equations. These equations are not yet ready to be evolved in time since they do not take into account any boundary conditions. However, the Jacobian of the discrete spatial operator in~\eqref{eq:discrete_euler_nobc} is a key component in any implicit timestepping scheme. The Jacobian can be derived using the following observations. Let $\fn, \gn : \mathbb{R}^n \to \mathbb{R}^n$ be vector-valued differentiable functions and define the diagonal matrices
\[
  \diagn{\fn} =
  \begin{pmatrix}
    f_1 &     &        &     \\
        & f_2 &        &     \\
        &     & \ddots &     \\
        &     &        & f_n \\
  \end{pmatrix}
  \text{ and }
  \diagn{\gn} =
  \begin{pmatrix}
    g_1 &     &        &     \\
        & g_2 &        &     \\
        &     & \ddots &     \\
        &     &        & g_n \\
  \end{pmatrix} \,.
\]
It is readily seen that the Jacobian $J_{\fn\gn}$ of the elementwise product $\fn\gn$ satisfies the following product rule:
\begin{equation}
  \label{eq:jacobian_product_rule}
  J_{\fn\gn} = \diagn{\fn}J_{\gn} + \diagn{\gn}J_{\fn} \,.
\end{equation}
Furthermore, the Jacobian is linear in the sense that for any $n$-by-$n$ matrices $A$ and $B$ we have
\begin{equation}
  \label{eq:jacobian_linear}
  J_{A\fn + B\gn} = AJ_{\fn} + BJ_{\gn} \,.
\end{equation}
Applying these Jacobian properties to $\euln$ yields the following proposition.

\begin{proposition}
  The Jacobian $J_{\euln}$ of the discrete operator $\euln$ in~\eqref{eq:discrete_euler_explicit} is
  \begin{equation*}
    J_{\euln} =
    \begin{pmatrix}
      \nablan_{\un} \euln_1 & \nablan_{\vn} \euln_1 & \nablan_{\pn} \euln_1 \\
      \nablan_{\un} \euln_2 & \nablan_{\vn} \euln_2 & \nablan_{\pn} \euln_2 \\
      \nablan_{\un} \euln_3 & \nablan_{\vn} \euln_3 & \nablan_{\pn} \euln_3
    \end{pmatrix}
  \end{equation*}
  where
  \begin{align*}
    \nablan_{\un} \euln_1 &= \frac{1}{2} \left(\diagn{\un}\Dx + \diagn{\Dx\un} + 2\Dx\diagn{\un} + \diagn{\vn}\Dy + \Dy\diagn{\vn}\right) \\
    \nablan_{\vn} \euln_1 &= \frac{1}{2} \left(\diagn{\Dy\un} + \Dy\diagn{\un}\right) \\
    \nablan_{\pn} \euln_1 &= \Dx \\
    \nablan_{\un} \euln_2 &= \frac{1}{2} \left(\diagn{\Dx\vn} + \Dx\diagn{\vn}\right) \\
    \nablan_{\vn} \euln_2 &= \frac{1}{2} \left(\diagn{\vn}\Dy + \diagn{\Dy\vn} + 2\Dy\diagn{\vn} + \diagn{\un}\Dx + \Dx\diagn{\un}\right) \\
    \nablan_{\pn} \euln_2 &= \Dy \\
    \nablan_{\un} \euln_3 &= \Dx \\
    \nablan_{\vn} \euln_3 &= \Dy \\
    \nablan_{\pn} \euln_3 &= \zeron \,.
  \end{align*}
\end{proposition}

\begin{proof}
  Let us find the first couple of terms in $\nablan_{\un} \euln_1$. The proofs for the other blocks are similar. Consider the first term $\frac{1}{2}\un \Dx \un$ of $\euln_1$. Let $\fn = \un$ and $\gn = \Dx \un$. Then $J_{\fn} = \In$ and $J_{\gn} = \Dx$. Hence, by~\eqref{eq:jacobian_product_rule},
  \begin{equation*}
    \nablan_{\un} \left( \frac{1}{2}\un \Dx \un \right) = \frac{1}{2} (\diagn{\un}\Dx + \diagn{\Dx \un}) \,.
  \end{equation*}
  Similarly, for the second term $\frac{1}{2} \Dx \un \un$ of $\euln_1$, we set $\fn = \un$ and $\gn = \un$. Then $J_{\fn} = J_{\gn} = \In$ and by~\eqref{eq:jacobian_product_rule} and~\eqref{eq:jacobian_linear} we get
  \begin{equation*}
    \nablan_{\un} \left( \frac{1}{2} \Dx \un \un \right) = \Dx \diagn{\un} \,.
  \end{equation*}

\end{proof}
