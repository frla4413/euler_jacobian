\section{The semi-discrete scheme}\label{sec:semi_discrete}
A brief introduction of the SBP-SAT technique is provided below and we recommend \cite{fernandez2014review,svard2014review} for extensive reviews. 

We discretize the domain $\Omega = [0,1]^2$ with $N+1$ and $M+1$ grid points; $x_i = i/N$, $i = 0,\dots, N$ and $y_j = j/M$, $j= 0,\dots,M$ and let $n = (N+1)(M+1)$ denote the total number of grid points. A scalar function $q = q(x,y)$ defined on $\Omega$ is thereby represented on the grid by $\qn = (q_{00}, \dots, q_{0M}, \dots q_{N0},\dots q_{NM})^\top$ where $q_{ij} = q(x_i,y_j)$. For the vector-valued function $\vecs{w} = (u,v,p)^\top$, the approximation is arranged as $\wn = (\un^\top,\vn^\top,\pn^\top)^\top$.
Let $\Dx = (P^{-1}_x Q_x) \otimes I_{M+1}$ and $\Dy = I_{N+1} \otimes (P^{-1}_y Q_y)$, where $\otimes$ denotes the Kronecker product. Then the approximations of the spatial derivatives are given by 
\[
  \Dx \un \approx \un_x,
  \quad 
  \Dy \un \approx \un_y
  \, .
\]
The matrices $P_{x,y}$ are diagonal and positive definite, so that $\Pn = P_x\otimes P_y$ forms a quadrature rule that defines the norm $\|\wn\|^2_{I_3\otimes\Pn} = \wn^\top (I_3\otimes\Pn) \wn \approx \iint_\Omega \vecs{w}^\top \vecs{w} d\Omega$. We have also introduced $I_{k}$, which is the identity matrix of size $k$. Moreover, the matrices $Q_{x,y}$ satisfy the SBP-property 
\begin{equation}
  Q_{x} + Q_{x}^\top = E_N - E_{0x}
  \quad 
  Q_{y} + Q_{y}^\top = E_M - E_{0y}
  \, ,
  \label{eq:sbp_property}
\end{equation}
where $E_{0x,y} = \text{diag}(1,0,0, \dots,0)$ and $E_{N,M} = \text{diag}(0,0,0,\dots,1)$ are matrices of appropriate sizes.

By using the notation above, the semi-discrete approximation of \eqref{eq:ins_continuous} becomes \cite{nordstrom2019energy}
\begin{equation}
  \label{eq:ins_semi_discrete}
  \Itn \wn_t  + \euln(\wn) = \satn(\wn) \, .
\end{equation}
The discrete spatial operator is given by
\begin{equation*}
\begin{aligned}
  \euln(\wn) &=
  \frac{1}{2}\left[\An (I_3\otimes \Dx)\wn + (I_3\otimes \Dx) \An\wn +
                   \Bn (I_3\otimes \Dy)\wn + (I_3\otimes \Dy) \Bn\wn\right]
  \\
  & - \epsilon \Itn [(I_3\otimes \Dx)^2 + (I_3\otimes \Dy)^2]\wn \, ,
\end{aligned}
\end{equation*}
and the block matrices are
\[
  \An = 
  \begin{pmatrix}
    \Un & \zeron & \In 
    \\
    \zeron  & \Un & \zeron
    \\
    \In & \zeron & \zeron
  \end{pmatrix}
  ,
  \quad 
  \Bn = 
  \begin{pmatrix}
    \Vn &  \zeron & \zeron 
    \\
    \zeron  & \Vn & \In
    \\
    \zeron & \In & \zeron
  \end{pmatrix}
  ,
  \quad
  \Itn = 
  \begin{pmatrix}
    \In &  \zeron & \zeron 
    \\
    \zeron  & \In & \zeron
    \\
    \zeron & \zeron & \zeron
  \end{pmatrix}
  \, ,
\]
where $\Un,\Vn \in \Rbb^{n\times n}$ are diagonal matrices holding $\un,\vn$, respectively. The matrices $\In$ and $\zeron$ are the identity and the zero matrix of size $n\times n$. Furthermore, $\satn(\wn)$ contains penalty terms that enforce the boundary conditions. 

The purpose of the SAT $\satn(\wn)$ is $i)$ to enforce the boundary conditions in \eqref{eq:boundary_conditions} and $ii)$ to stabilize the solution. 
One penalty term for each of the boundary conditions in \eqref{eq:boundary_conditions} will be constructed. Let $k\in\{W,E,S,N\}$. The SAT at boundary $k$ that enforces the boundary condition $H^k\vecs{w} = \g$ has the general form
\begin{equation}
  \satn^k(\wn) = (I_3\otimes\Pn^{-1})\Sigma^k(I_3\otimes\Pn^{k})(\Hn^k\wn - \vec{\gn})
  \, .
  \label{eq:penalty_general}
\end{equation}
In \eqref{eq:penalty_general}, $\Sigma^k$ is the penalty matrix to be determined for stability at boundary $k$. The quadratures are
\begin{equation*}
	\Pn^k = 
	\begin{cases}
    E_{0x}\otimes P_y 	\quad 	& \text{on the west boundary } (k = W)
		\\
    E_N\otimes P_y 	\quad  	& \text{on the east boundary } (k = E)
		\\
    P_x\otimes E_{0y} 	\quad  	& \text{on the south boundary } (k = S)
		\\
		P_x\otimes E_M 	\quad  	& \text{on the north boundary } (k = N)
    \, .
	\end{cases}
\end{equation*}

For the boundary conditions listed in \eqref{eq:boundary_conditions}, the penalty terms are
\begin{equation}
\begin{aligned}
   \satn^W(\wn) & = (I_3\otimes \Pn^{-1})
    \Sigma^W
   (I_3\otimes \Pn^W)
   \underbrace{
   \begin{pmatrix}
      \un - \gn_1
      \\
      \vn - \gn_2
      \\
      \un - \gn_1
   \end{pmatrix}
   }_{\Hn^W\wn - \vecs{\gn}}
   \\
   \satn^E(\wn) & = (I_3\otimes \Pn^{-1}) 
   \Sigma^E
   (I_3\otimes \Pn^E)
   \underbrace{
   \begin{pmatrix}
      \pn - \epsilon \Dx \un - \gn_3
      \\
      -\epsilon \Dx \vn - \gn_4
      \\
      \zeron
   \end{pmatrix}
   }_{\Hn^E\wn - \vecs{\gn}}
  \\
  \satn^S(\wn) & = (I_3\otimes \Pn^{-1})
  \Sigma^S(I_3\otimes \Pn^S)
   \underbrace{
   \begin{pmatrix}
      \un - \zeron
      \\
      \vn - \zeron
      \\
      \vn - \zeron
   \end{pmatrix}
   }_{\Hn^S\wn - \zeron}
   \\
  \satn^N(\wn) & = (I_3\otimes \Pn^{-1}) 
  \Sigma^N
   (I_3\otimes \Pn^N)
   \underbrace{
   \begin{pmatrix}
      - \epsilon \Dy \un - \gn_6
      \\
      \pn -\epsilon \Dy \vn - \gn_5
      \\
      \zeron
   \end{pmatrix}
   }_{\Hn^N\wn - \vecs{\gn}}
   \, ,
  \end{aligned}
\label{eq:penalty_terms}
\end{equation}
where 
\begin{align*}
   \Sigma^W & = 
   \begin{pmatrix}
      -\Un/2 + \epsilon \Dx^\top & \zeron & \zeron
      \\
      \zeron & -\Un/2 + \epsilon \Dx^\top & \zeron
      \\
      \zeron & \zeron & -\In
   \end{pmatrix}
   ,
   \quad
   &\Sigma^E & = (I_3\otimes \In)
   \\
   \Sigma^S & = 
    \begin{pmatrix}
      -\Vn/2 + \epsilon \Dy^\top & \zeron & \zeron
      \\
      \zeron & -\Vn/2 + \epsilon \Dy^\top & \zeron
      \\
      \zeron & \zeron & -\In 
   \end{pmatrix}
   ,
   \quad
   &\Sigma^N & = (I_3\otimes \In)
   \, .
\end{align*}
As an example, the south penalty term can be written as 
\begin{equation}
\begin{aligned}
  \satn^S(\wn) & = 
  %(I_3 \otimes \Pn^{-1})
  %\underbrace{
  %\begin{pmatrix}
  %   (-\Vn/2 + \epsilon \Dy^\top)\Pn^S \un
  %   \\
  %   (-\Vn/2 + \epsilon \Dy^\top)\Pn^S \vn
  %   \\
  %   -\Pn^S \vn
  %\end{pmatrix}
  %}_{\Sigma^S (I_3\otimes \Pn^s)\Hn^s \wn}
  %= 
  (I_3 \otimes \Pn^{-1})
  \begin{pmatrix}
     -\Vn\Pn^S\un/2 + \epsilon \Dy^\top\Pn^S\un
     \\
     -\Vn\Pn^S\vn/2 + \epsilon \Dy^\top\Pn^S\vn
     \\
     -\Pn^S \vn
  \end{pmatrix}
  \, ,
  \label{eq:SatS}
\end{aligned}
\end{equation}
which is a more convenient notation for the derivation of the Jacobian in \Cref{sec:jacobian}.

We will show in the following section that this specific choice of penalty matrices leads to nonlinear stability.
The total penalty term in \eqref{eq:ins_semi_discrete} becomes
\begin{equation}
  \satn(\wn) = \sum_{k\in\{W,E,S,N\}} \satn(\wn)^k
  \, .
  \label{eq:total_penalty}
\end{equation}

\subsection{Boundedness and Stability}
For completeness, we also show schematically how to obtain an energy estimate (again all details can be found in \cite{nordstrom2019energy}). Similarly to the continuous analysis, we omit all boundaries except for the south one. By mimicking the continuous path \cite{nordstrom2017roadmap}, we multiply \eqref{eq:ins_semi_discrete} by $2\wn^\top (I_3\otimes \Pn)$ from the left and use the SBP-property \eqref{eq:sbp_property} to get
\begin{equation}
  \frac{d}{dt}\|\wn\|^2_{\It\otimes \Pn} 
  + 2\epsilon \|\nabla \wn\|^2_{\It\otimes \Pn} = \BTn
  \, ,
  \label{eq:energy}
\end{equation}
where $\|\nabla \wn\|^2_{\It\otimes \Pn} = (I_3\otimes \Dx \wn)^\top (I_3\otimes \Pn)\Itn(I_3\otimes \Dx \wn) + (I_3\otimes \Dy \wn)^\top (I_3\otimes \Pn)\Itn(I_3\otimes \Dy \wn)$ is the dissipative volume term corresponding to the continuous one and 
\begin{equation}
\begin{aligned}
  \bm{BT} = & 
  \underbrace{
  \wn^\top(I_3 \otimes \Pn^S)\Bn\wn - 2\epsilon \wn^\top (I_3\otimes \Pn^S)\Itn (I_3 \otimes \Dy) \wn}_I
  \\
  & 
  \underbrace{
  + 
  2 \wn (I_3\otimes \Pn) \satn^S(\wn)}_{II}
  \label{eq:BT}
\end{aligned}
\end{equation}
contains all terms evaluated at the boundary. 

The semi-norm of the solution ($\|\wn\|^2_{\It\otimes \Pn}$)  is bounded if the right-hand side of \eqref{eq:energy} is non-positive. By expanding \eqref{eq:BT} and using the explicit form of $\satn^S(\wn)$ stated in \eqref{eq:penalty_terms}, we find
\begin{equation*}
\begin{aligned}
 \bm{BT} & = && 
 \underbrace{
 \vn^\top \Pn^S(\Un\un + \Vn \vn + 2\pn - 2 \epsilon  \Dy \vn ) -2 \epsilon\un^\top \Pn^S\Dy\un}_I
  \\
  &&&
  \underbrace{
  - 2\vn^\top \Pn^S(\Un \un/2 + \Vn \vn/2 + \pn^\top \vn - \epsilon \Dy \vn) 
  + 2\epsilon \un^\top \Pn^S \Dy \un}_{II}
  = 0
  \, ,
\end{aligned}
\end{equation*}
where term $I$ is obtained from the governing equation and term $II$ from the penalty term.
As in the continuous setting, the boundary terms vanish.
Integrating \eqref{eq:energy} in time (assuming homogeneous dissipative boundary conditions at all boundaries) leads to
\begin{equation*}
  \|\wn\|^2_{\It\otimes \Pn} (T)
  + 2\epsilon \int_0^T\|\nabla \wn\|^2_{\It\otimes \Pn} dt \le \|\fn\|^2_{\It\otimes \Pn}
  \, ,
\end{equation*}
which is the semi-discrete version of the estimate in \eqref{eq:estimate_continuous}.
