\section{Numerical experiments}%
\label{sec:numerical_experiments}
To evolve the system in time, we will use the implicit Backward Euler method. For an ordinary differential equation of the form
\begin{equation*}
 \Mn \soln_t + \Hn(\soln) = 0
 \, ,
\end{equation*}
where $\soln$ is a grid function and $\Mn$ is a constant matrix, the Backward Euler schemes becomes 
\begin{equation}
  \frac{\Mn (\soln^{i+1} - \soln^i)}{\Delta t} + \Hn(\soln^{i+1}) = 0
  \,.
  \label{eq:backward_euler}
\end{equation}
In~\eqref{eq:backward_euler}, $\Delta t$ is the size of the time step and the indices $i$ and $i+1$ are the solution at time level $i$ and $i+1$, respectively.

In order to obtain $\soln^{i+1}$, the system of nonlinear equations in~\eqref{eq:backward_euler} needs to be solved. One strategy is to first form the function 
\begin{equation}
  \Fn(\soln^{i+1}) = 
  \frac{\Mn (\soln^{i+1} - \soln^i)}{\Delta t} + \Hn(\soln^{i+1})
  \,.
  \label{eq:F_function}
\end{equation}
If we find a vector $\soln^*$ such that $\Fn(\soln^*) = 0$, then $\soln^{i+1} = \soln^*$. To solve~\eqref{eq:F_function}, we can employ Newton's method~\cite{quarteroni2010numerical}, which is described in Algorithm~\ref{alg:newton}. This allows us to solve a sequence of linear system of equations to obtain an approximation of $\soln^*$.
\begin{algorithm}
  \KwData{Initial guess, $\soln^0$, and tolerance $tol$}
  \KwResult{An approximation of $\soln^*$, where $\Fn(\soln^*) = 0$}
  \For{$i = 0,1,2, \dots$}{%
   solve $J_{\Fn}(\soln^i)\bm{h}^i   = - \Fn(\soln^i)$
   \\
   set   $\soln^{i+1}  = \soln^i + \bm{h}^i$

   \If{$\|\Fn(\soln^{i+1})\| < tol$}{%
    return $\soln^{i+1}$
 }
 }
 \caption{Newton's method.}%
 \label{alg:newton}
\end{algorithm}

For the Euler equations, $\Hn(\soln) = \euln(\soln) + \satn(\soln)$, where
\[
  \soln =
  \begin{bmatrix}
    \un & \vn & \pn
  \end{bmatrix}^\top
  \quad \text{and} \quad
  \Mn =
  \begin{pmatrix}
    \In    & \zeron & \zeron \\
    \zeron & \In    & \zeron \\
    \zeron & \zeron & \zeron
  \end{pmatrix}
  \,.
\]
Hence,~\eqref{eq:backward_euler} becomes
\begin{equation}
  \frac{1}{\Delta t} 
  \left(
    \begin{bmatrix}
      \un^{i+1} \\ \vn^{i+1} \\ \zeron
    \end{bmatrix}
    -
    \begin{bmatrix}
      \un^{i} \\ \vn^{i} \\ \zeron
    \end{bmatrix}
    \right)
    + \euln(\soln^{i+1}) + \satn(\soln^{i+1})
    = 0
  \label{eq:backward_euler_euler}
\end{equation}
and~\eqref{eq:F_function} turns into
\begin{equation}
  \Fn(\soln^{i+1}) = 
  \frac{1}{\Delta t} 
  \left(
    \begin{bmatrix}
      \un^{i+1} \\ \vn^{i+1} \\ \zeron
    \end{bmatrix}
    -
    \begin{bmatrix}
      \un^{i} \\ \vn^{i} \\ \zeron
    \end{bmatrix}
    \right)
    + \euln(\soln^{i+1}) + \satn(\soln^{i+1})
  \, .
  \label{eq:F_function_euler}
\end{equation}
Furthermore, $J_{\Hn}(\soln) = J_{\euln}(\soln) + J_{\satn}(\soln)$, which yields
\begin{equation*}
  J_{\Fn}(\soln) =  
  \frac{1}{\Delta t}
  \Mn
  +
  J_{\euln}(\soln)
  +
  J_{\satn}(\soln)
  \, ,
\end{equation*}
to be used in the Newton iterations. The explicit forms of $J_{\euln}(\soln)$ and $J_{\satn}(\soln)$ are found in~\ref{sec:discretization} and Section~\ref{sec:discrete_boundary_conditions}, respectively.

In all computations below, the initial guess is the solution from the previous time step and the tolerance is set to $10^{-12}$.

%\subsection{Stability analysis}
%Multiplying \eqref{eq:backward_euler} by $(\un^{i+1})^\top$, we get 
%\begin{equation*}
%  \|\un^{i+1}\|^2 = (\un^{i+1})^\top \un^{i} - \Delta t (\un^{i+1})^\top \Hn(\un^{i+1})
%  \, ,
%\end{equation*}
%where $\|\un\|^2 = \un^\top \un$. If $\un^\top \Hn(\un) \ge 0$ for all $\un$, then 
%\begin{equation*}
%   \|\un^{i+1}\|^2 \le (\un^{i+1})^\top \un^{i} \le \|\un^i\|\|\un^{i+1}\|
%   \Rightarrow 
%   \underline{
%   \underline{
%   \|\un^{i+1}\| \le \|\fn\|}}
%  \, ,
%\end{equation*}
%where $\fn$ is the initial data.

%For the Euler eqautions, the 

%The fully discrete form of the incompressible Euler equations~\eqref{eq:euler_nobc_split} becomes 
%\begin{equation}
%   \frac{1}{dt}\left[
%    \begin{pmatrix}
%      \un^{i+1} \\ \vn^{i+1} \\ \zeron
%    \end{pmatrix}
%    -
%    \begin{pmatrix}
%      \un^{i} \\ \vn^{i} \\ \zeron
%    \end{pmatrix}
%    \right]
%    + \euln(\un^{i+1},\vn^{i+1},\pn^{i+1}) = 0
%    %,
%    %\quad i = 1,\dots, N_t-1
%    \, ,
%    \label{eq:backward_euler}
%\end{equation}
%for $i = 1,\dots N_t-1$, which must be solved to advance in time. As can be seen from~\eqref{eq:backward_euler}, no initial condtion for the pressure is required.

\subsection{The order of accuracy}
The method of manufactured solution is used to verify the implementation. For the SBP-operators SBP21 and SBP42, the expected orders of accuracy are 2 and 3, respectively~\cite{svard2014review}. The manufactured solution is 
\begin{equation*}
  \begin{aligned}
    u & = 1 + 0.5 \sin(2\pi x-t)\sin(2\pi y-t)
    \\
    v & = 1 + 0.5 \cos(2\pi x-t)\cos(2\pi y-t)
    \\
    p & = 0.5 \sin(2\pi x-t)\cos(2\pi y-t)
 \end{aligned}
\end{equation*}
and the domain is illustrated in Figure~\ref{fig:bump_flow}. Since the west and south sides of the domain are inflow boundaries, both $w_n$ and $w_t$ are specified (see Section~\ref{sec:continuous_inflow} and Section~\ref{sub:inflow} for details). At the east and north sides, which are outflow boundaries, only the pressure is prescribed (see Section~\ref{sec:pressure_outflow} and Section~\ref{sub:pressure_outflow} for details).

The step size is chosen to be $\Delta t = 10^{-4}$ and the computations are terminated at $t = 1$. Then, we compute the pointwise error, $\en$ and its $L_2$-norm $\|\en\|_{\Omega_h}$. The convergence rate is given by $r = \log(\|\en\|_{i}/\|\en\|_{j})/\log((j-1)/(i-1))$, where $i$ and $j$ refer to the number of grid points in both spatial dimensions. The results are presented in Table~\ref{tab:convergence_table} and agree well with theory.
\begin{table}
\centering
\setlength{\tabcolsep}{12pt}
\begin{tabular}{c| cc | cc cc}
 \hline
 \hline
 operator
 & \multicolumn{2}{c|}{SBP21}
 & \multicolumn{2}{c}{SBP42}
 \\
\hline
 \hline
N & $\|\en\|$ & $r$ &$\|\en\|$ & $r$
\\
\hline
11 & 1.60e+00 & --   & 5.19e-01 & --
\\                   
21 & 1.84e-01 & 3.12 & 8.93e-02 & 2.54 
\\                   
41 & 4.25e-02 & 2.12 & 1.32e-02& 2.76
\\                   
61 & 2.84e-02 & 2.05 & 3.92e-03& 2.98  
\\
81& 1.03e-02 & 2.02  & 1.67e-03 & 2.97  
\\
101& 6.58e-03 & 2.01 & 8.61e-04 & 2.95 
\\
\hline
Theoretical && 2 && 3
\end{tabular}
\caption{Error and convergence rate.}%
\label{tab:convergence_table}
\end{table}

\FloatBarrier%
