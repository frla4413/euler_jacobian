\section{Exact computation of the Jacobian}\label{sec:jacobian}
In this section, we will explicitly compute the Jacobian of $\euln$ and $\satn$ in \eqref{eq:ins_semi_discrete}. Let $\hn :\Rbb^{n} \to \Rbb^{n}$, where $n = (N+1)(M+1)$ is the total number of grid points, be a differentiable vector function. For a given vector $\un = (u_{00}, \dots u_{NM})^\top \in \Rbb^{n}$, $\hn$ outputs the vector $\hn(\un) = (h_{00}, \dots h_{NM})^\top \in \Rbb^{n}$. The Jacobian matrix $J_{\hn} \in \Rbb^{n\times n}$ of $\hn$ is given by
\[
  J_{\hn} =  
  \begin{pmatrix}
     \frac{\partial h_{00}}{\partial u_{00}} & \dots & 
     \frac{\partial h_{00}}{\partial u_{NM}} 
     \\
     \vdots  &\ddots & \vdots
     \\
     \frac{\partial h_{NM}}{\partial u_{00}} & \dots 
     & \frac{\partial h_{NM}}{\partial u_{NM}} 
  \end{pmatrix}
  \, .
\]

We will first derive the Jacobian of the different terms in $\euln(\wn)$ and at the end, add the terms and state the complete result. To start, consider the vector function 
\[
  \hn(\un) 
  =
  \begin{pmatrix}
      h_{00} \\ \vdots \\ h_{NM} 
  \end{pmatrix}
  = 
  \begin{pmatrix}
      u_{00} \\ \vdots \\ u_{NM} 
  \end{pmatrix}
  =
  \un
  \, .
\]
Since
{\footnotesize
\begin{align*}
  &\frac{\partial h_{00}}{\partial u_{00}} = 1 %\quad 
  &\frac{\partial h_{00}}{\partial u_{01}} = 0 %\quad 
  &&\frac{\partial h_{00}}{\partial u_{02}} = 0 %\quad 
  &&\dots 
  &&\frac{\partial h_{00}}{\partial u_{NM}} = 0 %\quad 
  \\
  &\frac{\partial h_{01}}{\partial u_{00}} = 0 %\quad 
  &\frac{\partial h_{01}}{\partial u_{01}} = 1 %\quad 
  &&\frac{\partial h_{01}}{\partial u_{02}} = 0 %\quad 
  &&\dots 
  &&\frac{\partial h_{01}}{\partial u_{NM}} = 0 %\quad 
  \\
  & & & \vdots 
  \\
  &\frac{\partial h_{NM}}{\partial u_{00}} = 0 %\quad 
  &\frac{\partial h_{NM}}{\partial u_{01}} = 0 %\quad 
  &&\frac{\partial h_{NM}}{\partial u_{02}} = 0 %\quad 
  &&\dots 
  &&\frac{\partial h_{NM}}{\partial u_{NM}} = 1 %\quad 
  \, ,
\end{align*}
}
the Jacobian of $\hn(\un) = \un$ becomes $J_{\un} = \In$.
%\[
%  J_{\un} =  
%  \begin{pmatrix}
%    \frac{\partial h_{00}}{\partial u_{00}} & \dots & 
%    \frac{\partial h_{00}}{\partial u_{NM}} 
%     \\
%     \vdots  &\ddots & \vdots
%     \\
%     \frac{\partial h_{NM}}{\partial u_{00}} & \dots & 
%     \frac{\partial h_{NM}}{\partial u_{NM}} 
%  \end{pmatrix}
%  =
%  \begin{pmatrix}
%    1 & 0 & \dots 
%    \\
%    0 & 1 & 0 & \dots
%    \\
%    & & \ddots 
%    \\
%    & & 0 & 1
%  \end{pmatrix}
%  = \In
%  \, .
%\]
Now let
\begin{equation*}
\begin{aligned}
  \hn(\un) & =  
  \begin{pmatrix}
      h_{00} \\ \vdots \\ h_{NM} 
  \end{pmatrix}
  =
  \Dx \un
  = 
  \begin{pmatrix}
    D_{0,0} & \dots & D_{0,NM}
    \\
    \vdots  &\ddots & \vdots
    \\
    D_{NM,0} & \dots & D_{NM,MN}
  \end{pmatrix}
  \begin{pmatrix}
    u_{00} \\ \vdots \\u_{NM}
  \end{pmatrix}
  \\
  & = 
  \begin{pmatrix}
    D_{0,0}u_{00} + \dots + D_{0,NM}u_{NM}
    \\
    \ \vdots  \quad \quad 
    \\
    D_{NM,0} u_{00} + \dots + D_{NM,MN}u_{NM}
  \end{pmatrix}
  \, .
\end{aligned}
\end{equation*}
Then, in the same way
{\footnotesize
\begin{align*}
  &\frac{\partial h_{00}}{\partial u_{00}} = D_{0,0} %\quad 
  &\frac{\partial h_{00}}{\partial u_{01}} = D_{0,1} %\quad 
  &&\frac{\partial h_{00}}{\partial u_{02}} = D_{0,2} %\quad 
  &&\dots 
  &&\frac{\partial h_{00}}{\partial u_{NM}} = D_{0,NM} %\quad 
  \\
  &\frac{\partial h_{01}}{\partial u_{00}} = D_{1,0} %\quad 
  &\frac{\partial h_{01}}{\partial u_{01}} = D_{1,1} %\quad 
  &&\frac{\partial h_{01}}{\partial u_{02}} = D_{1,2} %\quad 
  &&\dots 
  &&\frac{\partial h_{01}}{\partial u_{NM}} = D_{1,NM} %\quad 
  \\
  & & & \vdots 
  \\
  &\frac{\partial h_{NM}}{\partial u_{00}} = D_{NM,0} %\quad 
  &\frac{\partial h_{NM}}{\partial u_{01}} = D_{NM,1} %\quad 
  &&\frac{\partial h_{NM}}{\partial u_{02}} = D_{NM,2} %\quad 
  &&\dots 
  &&\frac{\partial h_{NM}}{\partial u_{NM}} = D_{NM,NM} %\quad 
  \, .
\end{align*}
}
Thus, $J_{\Dx\un} = \Dx$.
%\[
%   =  
%  \begin{pmatrix}
%    D_{0,0} & \dots & D_{0,NM}
%    \\
%    \vdots  &\ddots & \vdots
%    \\
%    D_{NM,0} & \dots & D_{NM,MN}
%  \end{pmatrix}
%  = \Dx
%  \, .
%\]

To derive the Jacobian of the nonlinear term, $\Un\Dx\un$, we let
\[
  \hn(\un) = \Un\Dx \un =
  \begin{pmatrix}
    u_{00}[D_{0,0}u_{00} + \dots + D_{0,NM}u_{NM}]
    \\
    \vdots 
    \\
    u_{NM}[D_{NM,0} u_{00} + \dots + D_{NM,MN}u_{NM}]
  \end{pmatrix}
  = 
  \begin{pmatrix}
    u_{00}(\Dx \un)_{00}
    \\
    \vdots 
    \\
    u_{NM}(\Dx \un)_{NM}
  \end{pmatrix}
  \, .
\]
By using the product rule, we get that
{\footnotesize
\begin{align*}
  &\frac{\partial h_{00}}{\partial u_{00}} = u_{00} D_{0,0} + (\Dx \un)_{00}
  &&\frac{\partial h_{00}}{\partial u_{01}} = u_{00} D_{0,1} 
  & \dots\quad 
  &\frac{\partial h_{00}}{\partial u_{NM}} = u_{00}D_{0,NM}
  \\
  &\frac{\partial h_{01}}{\partial u_{00}} = u_{01} D_{1,0}
  &&\frac{\partial h_{01}}{\partial u_{01}} = u_{01} D_{1,1} + (\Dx \un)_{01}
  & \dots \quad
  &\frac{\partial h_{01}}{\partial u_{NM}} = u_{00}D_{0,NM}
  \\
  & \quad \quad \vdots 
  \\
  &\frac{\partial h_{NM}}{\partial u_{00}} = u_{NM} D_{NM,0}
  &&\frac{\partial h_{NM}}{\partial u_{01}} = u_{NM} D_{NM,1}
  & \dots\quad
  &\frac{\partial h_{NM}}{\partial u_{NM}} = u_{NM} D_{NM,NM} + (\Dx \un)_{NM}
  \, .
\end{align*}
}
Hence,
\begin{equation*}
\begin{aligned}
  J_{\Un\Dx\un} & = 
  \begin{pmatrix}
    u_{00} D_{0,0} + (\Dx \un)_{00}& u_{00} D_{0,1} & \dots & u_{00}D_{0,NM}
    \\
    u_{01} D_{1,0} & u_{01} D_{1,1} + (\Dx \un)_{01} & \dots & u_{01}D_{1,NM}
    \\
    && \vdots
    \\
    u_{NM} D_{NM,0} & u_{NM} D_{NM,1} & \dots & u_{NM}D_{NM,NM} + (\Dx \un)_{NM}
  \end{pmatrix}
  \\
  & = 
  \underbrace{
  \begin{pmatrix}
    u_{00} D_{0,0} & u_{00} D_{0,1} & \dots & u_{00}D_{0,NM}
    \\
    u_{01} D_{1,0} & u_{01} D_{1,1}  & \dots & u_{01}D_{1,NM}
    \\
    && \vdots
    \\
    u_{NM} D_{NM,0} & u_{NM} D_{NM,1} & \dots & u_{NM}D_{NM,NM}
  \end{pmatrix}
  }_{\Un \Dx}
  \\
  & +
  \underbrace{
  \begin{pmatrix}
    (\Dx \un)_{00} & 0 & \dots & 0
    \\
    0 & (\Dx \un)_{01} & \dots & 0
    \\
    && \vdots
    \\
    0 &  0 & \dots & (\Dx \un)_{NM}
  \end{pmatrix}
  }_{\diagn{\Dx \un}}
  = \Un \Dx + \diagn{\Dx \un}
  \, ,
\end{aligned}
\end{equation*}
where $\diagn{\Dx \un} = \text{diag}(\Dx \un)$.

In a similar manner, for
\[  
  \hn(\un) = \Dx \Un\un =
  \begin{pmatrix}
    D_{0,0}u_{00}^2 + \dots + D_{0,NM}u_{NM}^2
    \\
    \vdots 
    \\
    D_{NM,0} u_{00}^2 + \dots + D_{NM,MN}u_{NM}^2
  \end{pmatrix}
\]
we get that 
{\footnotesize
\begin{align*}
  &\frac{\partial h_{00}}{\partial u_{00}} = 2 D_{0,0}u_{00}
  &\frac{\partial h_{00}}{\partial u_{01}} = 2 D_{0,1}u_{01}
  \quad
  \dots 
  \quad 
  &\frac{\partial h_{00}}{\partial u_{NM}} = 2 D_{0,NM}u_{NM}
  \\
  &\frac{\partial h_{01}}{\partial u_{00}} = 2 D_{1,0}u_{00}
  &\frac{\partial h_{01}}{\partial u_{01}} = 2 D_{1,1}u_{01}
  \quad
  \dots 
  \quad 
  &\frac{\partial h_{01}}{\partial u_{NM}} = 2 D_{1,NM}u_{NM}
  \\
  & \quad \quad \vdots 
  \\
  &\frac{\partial h_{NM}}{\partial u_{00}} = 2 D_{NM,0}u_{00}
  &\frac{\partial h_{NM}}{\partial u_{01}} = 2 D_{NM,1}u_{01}
  \quad
  \dots 
  \quad 
  &\frac{\partial h_{01}}{\partial u_{NM}} = 2 D_{NM,NM}u_{NM}
  \, .
\end{align*}
}
Hence, $J_{\Dx\Un\un} = 2 \Dx \Un$.
%\begin{align*}
%  J_{\Dx\Un\un} & = 2
%  \begin{pmatrix}
%    D_{0,0} u_{00} & D_{0,1} u_{01} & \dots & D_{0,NM} u_{NM}
%    \\
%    D_{1,0} u_{00} & D_{1,1} u_{01} & \dots & D_{1,NM} u_{NM}
%    \\
%    && \vdots 
%    \\
%    D_{NM,0} u_{00} & D_{NM,1} u_{01} & \dots & D_{NM,NM} u_{NM}
%  \end{pmatrix}
%  =
%  2 \Dx \Un
%  \, .
%\end{align*}
To summarize, we have shown that
\begin{equation}
  J_{\un} = \In, \quad J_{\Dx \un } = \Dx, \quad 
  J_{\Un \Dx \un} = \Un \Dx + \diagn{\Dx\un}, \quad 
  J_{\Dx \Un \un} = 2 \Dx \Un
  \, .
  \label{eq:jacobians}
\end{equation}

%\textcolor{blue}{
%\begin{remark}
%For completeness, we also state the Jacobian of $\hn(\un) = \Dx (1/\un)$, where the division should be interpreted element-wise. Such terms are for example present in discretizations of the compressible Navier-Stokes equations. By following the same procedure presented above, we get that $J_{\Dx(1/\un) }= -\Dx \diagn{1/(\Un\un)}$.
%\end{remark}
%}

\subsection{The Jacobian of the spatial operator}%
\label{sub:the_jacobian_of_the_spatial_operator}
Having established these building blocks, we next consider the terms in $\euln(\wn)$. Since these terms have a block structure, their Jacobians will have that as well. Let $\hn^1,\hn^2,\hn^3: \Rbb^{3n} \to \Rbb^n$ be differentiable functions of $\wn$ and define $\tilde{\hn} : \Rbb^{3n}\to \Rbb^{3n}$ given by
\[
   \tilde{\hn}(\wn) = 
   \begin{pmatrix}
      \hn^1(\wn) \\ \hn^2(\wn) \\ \hn^3(\wn)
   \end{pmatrix}
   \, .
\]
Since 
\[
    \hn^1  = 
    \begin{pmatrix}
        h^1_{00} \\ h^1_{01} \\ \vdots \\ h^1_{NM}
    \end{pmatrix}
    \quad 
    \text{and}
    \quad
    \wn = 
    \begin{pmatrix}
        \un \\ \vn \\ \pn
    \end{pmatrix}
\]
it follows that 
\begin{align*}
    J_{h^1_{00}} = 
    \frac{\partial h^1_{00}}{\partial \wn} 
    & =
    \begin{pmatrix}
       \frac{\partial h^1_{00}}{\partial w_{00}} 
       & 
       \dots 
       &
       \frac{\partial h^1_{00}}{\partial w_{3NM}} 
    \end{pmatrix}
    =
    \begin{pmatrix}
        \frac{\partial h^1_{00}}{\partial \un} 
        & 
        \frac{\partial h^1_{00}}{\partial \vn}
        &
        \frac{\partial h^1_{00}}{\partial \pn}
    \end{pmatrix} \in \Rbb^{1\times 3n}
\end{align*}
and similarly for every element in $\hn^1$. Therefore, the Jacobian of $\hn^1$ can be expressed as
\[
    J_{\hn^1} = 
    \frac{\partial \hn^1}{\partial \wn} 
    =
    \begin{pmatrix}
      \frac{\partial h^1_{00}}{\partial \un} & 
      \frac{\partial h^1_{00}}{\partial \vn} & 
      \frac{\partial h^1_{00}}{\partial \pn} 
      \\
      \frac{\partial h^1_{01}}{\partial \un} & 
      \frac{\partial h^1_{01}}{\partial \vn} & 
      \frac{\partial h^1_{01}}{\partial \pn} 
      \\
      \vdots  & \vdots & \vdots
      \\
      \frac{\partial h^1_{NM}}{\partial \un} & 
      \frac{\partial h^1_{NM}}{\partial \vn} & 
      \frac{\partial h^1_{NM}}{\partial \pn} 
    \end{pmatrix}
    = 
    \begin{pmatrix}
      \frac{\partial \hn^1}{\partial \un} & 
      \frac{\partial \hn^1}{\partial \vn} & 
      \frac{\partial \hn^1}{\partial \pn} 
    \end{pmatrix}
    \in \Rbb^{n\times 3n}
    \, .
\]
The same holds for $\hn^2$ and $\hn^3$. Thus, the Jacobian of $\tilde{\hn}$ is given by
\[
  J_{\tilde{\hn}} =
  \begin{pmatrix}
    \frac{\partial \hn^1 }{\partial \un} &
    \frac{\partial \hn^1 }{\partial \vn} & 
    \frac{\partial \hn^1 }{\partial \pn}
    \\
    \frac{\partial \hn^2 }{\partial \un} &
    \frac{\partial \hn^2 }{\partial \vn} & 
    \frac{\partial \hn^2 }{\partial \pn}
    \\
    \frac{\partial \hn^3 }{\partial \un} &
    \frac{\partial \hn^3 }{\partial \vn} & 
    \frac{\partial \hn^3 }{\partial \pn}
  \end{pmatrix}
  \in \Rbb^{3n\times 3n}
\]
and each block in $J_{\tilde{\hn}}$ is of size $n\times n$. 

For the first term in $\euln(\wn)$, $\An (I_3 \otimes \Dx) \wn$, we get 
\[
  \tilde{\hn}(\wn) = 
  \begin{pmatrix}
    \hn^1(\wn) \\ \hn^2(\wn) \\ \hn^3(\wn) 
  \end{pmatrix}
  =
  \An (I_3 \otimes \Dx) \wn = 
  \begin{pmatrix}
    \Un \Dx \un  + \Dx \pn
    \\
    \Un \Dx \vn
    \\
    \Dx \un
  \end{pmatrix}
  =
  \begin{pmatrix}
    \Un \Dx \un  + \Dx \pn
    \\
    \diagn{\Dx \vn}\un 
    \\
    \Dx \un
  \end{pmatrix}
  \, .
\]
The last identities are useful when deriving $J_{\An (I_3 \otimes \Dx) \wn}$. By using \eqref{eq:jacobians}, we get that
\begin{align*}
  \frac{\partial\hn^1}{\partial \un}  & = \Un \Dx + \diagn{\Dx\un}
  \quad
  & \frac{\partial\hn^1}{\partial \vn} & = \zeron
  \quad
  & \frac{\partial\hn^1}{\partial \pn} & = \Dx
  \\
  \frac{\partial\hn^2}{\partial \un}  & = \diagn{\Dx\vn} 
  \quad
  &\frac{\partial\hn^2}{\partial \vn}  & = \Un \Dx
  \quad
  &\frac{\partial\hn^2}{\partial \pn}  & = \zeron
  \\
  \frac{\partial\hn^3}{\partial \un}  & = \Dx
  \quad
  &\frac{\partial\hn^3}{\partial \vn}  & = \zeron
  \quad
  &\frac{\partial\hn^3}{\partial \pn}  & = \zeron
  \, .
\end{align*}
Thus,
\[
  J_{\An (I_3 \otimes \Dx) \wn}
  =
  \begin{pmatrix}
     \Un \Dx + \diagn{\Dx\un} & \zeron & \Dx
     \\
     \diagn{\Dx \vn} & \Un \Dx & \zeron
     \\
     \Dx & \zeron & \zeron
  \end{pmatrix}
  \, .
\]
Likewise for the second term in $\euln(\wn)$, $(I_3 \otimes \Dx) \An \wn$, note that
\[
  (I_3 \otimes \Dx) \An \wn = 
  \begin{pmatrix}
    \Dx \Un \un  + \Dx \pn
    \\
    \Dx \Un \vn
    \\
    \Dx \un
  \end{pmatrix}
  =
  \begin{pmatrix}
    \Dx \Un \un  + \Dx \pn
    \\
    \Dx \Vn \un
    \\
    \Dx \un
  \end{pmatrix}
  \, ,
\]
where we have used that $\Vn \un  = \Un \vn$. Hence, 
\[
  J_{(I_3 \otimes \Dx) \An \wn} =
  \begin{pmatrix}
     2 \Dx \Un & \zeron & \Dx
     \\
     \Dx \Vn & \Dx \Un & \zeron
     \\
     \Dx & \zeron & \zeron
  \end{pmatrix}
  \, .
\]
The next two terms in $\euln(\wn)$ are treated in a similar manner and we get that
\begin{align*}
  J_{\Bn (I_3 \otimes \Dy) \wn} = &
  \begin{pmatrix}
     \Vn \Dy &  \diagn{\Dy\un} & \zeron
     \\
     \zeron & \Vn \Dy + \diagn{\Dy \vn} & \Dy
     \\
      \zeron & \Dy & \zeron
  \end{pmatrix}
  \\
  J_{(I_3 \otimes \Dy)\Bn\wn} = &
  \begin{pmatrix}
     \Dy\Vn &  \Dy\Un & \zeron
     \\
     \zeron & 2\Dy\Vn  & \Dy
     \\
      \zeron & \Dy & \zeron
  \end{pmatrix}
  \, .
\end{align*}
Finally, the contribution to the Jacobian of the linear viscous terms simply becomes
\[
  J_{-\epsilon\Itn [(I_3\otimes \Dx)^2 +(I_3\otimes \Dy)^2]\wn} =
  -
  \begin{pmatrix}
     \epsilon (\Dx^2 + \Dy^2) & \zeron & \zeron
     \\
     \zeron & \epsilon (\Dx^2 + \Dy^2) & \zeron
     \\
     \zeron & \zeron & \zeron
  \end{pmatrix}
  \, .
\]

Adding all terms proves the following proposition, which is the first of the two main results of this paper.

\begin{proposition}
  The Jacobian $J_{\euln}$ of the discrete operator $\euln$ in~\eqref{eq:ins_semi_discrete} is
  \begin{equation}
    J_{\euln} =
    \begin{pmatrix}
      J_{11} & \frac{1}{2} \left(\diagn{\Dy\un} + \Dy\Un\right)& \Dx \\
       \frac{1}{2} \left(\diagn{\Dx\vn} + \Dx\Vn\right) & J_{22} & \Dy \\
      \Dx & \Dy & \zeron \\
    \end{pmatrix}
    \label{eq:J_L}
  \end{equation}
  where
  \begin{align*}
    J_{11} &= \frac{1}{2} \left(\Un\Dx + \diagn{\Dx\un} + 2\Dx\Un + \Vn\Dy + \Dy\Vn\right)  - \epsilon (\Dx^2 + \Dy^2)\\
    J_{22} &= \frac{1}{2} \left(\Vn\Dy + \diagn{\Dy\vn} + 2\Dy\Vn + \Un\Dx + \Dx\Un\right)  - \epsilon (\Dx^2 + \Dy^2)
    \, .
  \end{align*}
  \label{prop:ins}
\end{proposition}

\subsection{The Jacobian of the penalty terms}
By following the procedure presented above, we next derive the Jacobian for $\satn(\wn)$. To start, we rewrite $\satn^S(\wn)$ as
\begin{equation}
  \satn^S(\wn) = 
  \begin{pmatrix}
    \satn_1^S \\
    \satn_2^S \\
    \satn_3^S \\
  \end{pmatrix}
  =
  (I_3\otimes \Pn^{-1}) 
  \begin{pmatrix}
    -\Vn\Pn^S \un/2 + \epsilon \Dy^\top \Pn^S \un 
    \\
     -\Vn\Pn^S\vn /2+ \epsilon \Dy^\top \Pn^S \vn
    \\
    - \Pn^S\vn
  \end{pmatrix}
  \in \Rbb^{3n}
  \, .
  \label{eq:sat_s}
\end{equation}
The Jacobian of $\satn^S(\wn)$ is
\[
  J_{\satn^S} = 
  \begin{pmatrix}
    \frac{\partial \satn_1^S}{\partial \un}
    & 
    \frac{\partial \satn_1^S}{\partial \vn}
    &
    \frac{\partial \satn_1^S}{\partial \pn}
    \\
    \frac{\partial \satn_2^S}{\partial \un}
    & 
    \frac{\partial \satn_2^S}{\partial \vn}
    &
    \frac{\partial \satn_2^S}{\partial \pn}
    \\
    \frac{\partial \satn_3^S}{\partial \un}
    & 
    \frac{\partial \satn_3^S}{\partial \vn}
    &
    \frac{\partial \satn_3^S}{\partial \pn}
  \end{pmatrix}
  \in \Rbb^{3n\times 3n}
  \, .
\]
The first block in $J_{\satn^S}$ becomes
{\small
\begin{align*}
  \frac{\partial \satn_1^S}{\partial \un} & =
  \Big(-
  \underbrace{\frac{\partial}{\partial \un}
  \Pn^{-1}\Vn\Pn^S \un/2}_{= \Pn^{-1}\Vn\Pn^S/2}
  + 
  \underbrace{
  \frac{\partial}{\partial \un}
  \epsilon \Pn^{-1}\Dy^\top \Pn^S\un}_{= \epsilon\Pn^{-1} \Dy^\top \Pn^S} \Big)
  =
  \Pn^{-1}
  \left(-\Vn/2 + \epsilon \Dy^\top\right)\Pn^S 
  \, .
\end{align*}
}
Since $\Pn^S$ is diagonal, we have $\Vn \Pn^s \un = \Un \Pn^S \vn$ and the second block is
{\small
\begin{align*}
  \frac{\partial \satn_1^S}{\partial \vn} & =
  \Big(-
  \underbrace{\frac{\partial}{\partial \vn}
  \Pn^{-1}\Un\Pn^S \vn/2}_{= \Pn^{-1}\Un \Pn^S /2}
  + 
  \underbrace{
  \frac{\partial}{\partial \vn}\epsilon \Pn^{-1}\Dx^\top \Pn^S\un}_{= \zeron} \Big)
  = -\Pn^{-1}\Un\Pn^S /2
  \, .
\end{align*}
}
Note that $\satn^S$ does not depend on $\pn$ and also that $\satn_2^S$ and $\satn_3^S$ are both independent of $\un$. Hence, the remaining non-zero blocks of $J_{\satn^S}$ are
\begin{align*}
   \frac{\partial \satn^W_2}{\partial \vn} &= \Pn^{-1}(-\Vn + \epsilon \Dy^\top)\Pn^S
   ,\quad  
   & \frac{\partial \satn_3^W}{\partial \vn} & = -\Pn^{-1}\Pn^S
   \, ,
\end{align*}
where we have used that $\Vn \Pn^s \vn = \Pn^s \Vn \vn$. 
Therefore, 
\[
    J_{\satn^S} = 
    (I_3 \otimes \Pn^{-1})
    \begin{pmatrix}
        -\Vn/2 + \epsilon \Dy^\top & -\Un/2 & \zeron
        \\
        \zeron  & -\Vn + \epsilon \Dy^\top & \zeron
        \\
        \zeron & -\In & \zeron
    \end{pmatrix}
    (I_3 \otimes \Pn^S)
    \, .
\]

For non-homogeneous boundary conditions, the boundary data $\gn$ will affect the Jacobian if the SAT:s are nonlinear with respect to $\wn$. We illustrate this by considering $\satn^W_1$ (the first block in $\satn^W$), which we rewrite in a similar manner as we did for $\satn^S_1$ in \eqref{eq:sat_s} and get
\begin{align*}
 \satn^W_1(\wn) & =  \Pn^{-1}(-\Un/2 + \epsilon\Dx^\top)\Pn^W(\un - \gn_1)
 \\
  & = -\Pn^{-1}\Un\Pn^W(\un - \gn_1)/2 
  + \epsilon\Pn^{-1}\Dx^\top\Pn^W(\un - \gn_1)
             \, .
\end{align*}
Note that the terms $-\Pn^{-1}\Un\Pn^W(\un - \gn_1)/2$ and $\epsilon \Pn^{-1}\Dx^\top\Pn^W(\un - \gn_1)$ are nonlinear and linear with respect to $\wn$ (via $\un$), respectively. The Jacobian to the linear term simply becomes
\begin{align*}
   \frac{\partial}{\partial \un}\left(\epsilon \Pn^{-1}\Dx^\top\Pn^W(\un - \gn_1) \right) 
   = 
   \underbrace{
   \frac{\partial}{\partial \un}\left(\epsilon \Pn^{-1}\Dx^\top\Pn^W\un\right) 
   }_{= \epsilon \Pn^{-1}\Dx^\top\Pn^W}
   -
   \underbrace{
   \frac{\partial}{\partial \un}\left(\epsilon \Pn^{-1}\Dx^\top\Pn^W\gn_1\right) 
   }_{ = 0}
   \, .
\end{align*}
For the nonlinear term, we use that $\Un \Pn^W = \Pn^W \Un$ and $\Un \gn_1 = \diagn{\gn_1}\un$, which yield
\begin{align*}
   -\frac{\partial}{\partial \un}\left(\Pn^{-1}\Pn^W\Un(\un - \gn_1)/2\right) 
   & = 
   \underbrace{
   -\frac{\partial}{\partial \un}\left(\Pn^{-1}\Pn^W\Un\un)/2\right)}_
   {= -\Pn^{-1}\Pn^W\Un} 
   +
   \underbrace{
   \frac{\partial}{\partial \un}\left(\Pn^{-1}\Pn^W\diagn{\gn_1}\un)/2\right)
   }_{=\Pn^{-1}\Pn^W \diagn{\gn_1}/2} 
   \\
   & = \Pn^{-1}(-\Un + \diagn{\gn_1}/2)\Pn^W
\end{align*}
Since $\satn_1^W$ is independent of both $\vn$ and $\pn$, its Jacobian becomes
\[
   J_{\satn^W_1}(\wn) = 
   \begin{pmatrix}
     \Pn^{-1}(-(\Un - \diagn{\gn_1}/2) + \epsilon \Dx^\top)\Pn^W  
     & \zeron & \zeron
   \end{pmatrix}
   \in \Rbb^{n\times 3n}
\]

The Jacobian of the other penalty terms are derived in a similar manner and we have therefore proved the second main result of this paper.

\begin{proposition}
The Jacobian of the total penalty term \eqref{eq:total_penalty} is
\begin{equation}
  J_{\satn} (\wn) = \sum_{k\in\{W,E,S,N\}} J_{\satn^k}(\wn)
  \, ,
  \label{eq:jacobian_penalty}
\end{equation}
where 
\begin{equation*}
\begin{aligned}
   J_{\satn^W}(\wn) & = (I_3\otimes \Pn^{-1})
  \begin{pmatrix}
   -(\Un-\diagn{\gn_1}/2) + \epsilon \Dx^\top & \zeron & \zeron
   \\
   -(\Vn-\diagn{\gn_2})/2 & -\Un/2 + \epsilon \Dx^\top & \zeron 
   \\
   -\In & \zeron & \zeron
  \end{pmatrix}
  (I_3\otimes \Pn^W)
  \\
  J_{\satn^E}(\wn) & = (I_3\otimes \Pn^{-1}\Pn^E) 
  \begin{pmatrix}
   - \epsilon \Dx & \zeron & \In
   \\
   \zeron & - \epsilon \Dx & \zeron 
   \\
   \zeron & \zeron & \zeron
  \end{pmatrix}
  \\
  J_{\satn^S}(\wn) & = (I_3\otimes \Pn^{-1}\Pn^S) 
  \begin{pmatrix}
   -\Vn/2 + \epsilon \Dy^\top& -\Un/2 & \zeron
   \\
   \zeron & -\Vn + \epsilon \Dy^\top & \zeron 
   \\
   \zeron & -\In & \zeron
  \end{pmatrix}
  (I_3\otimes \Pn^S)
  \\
  J_{\satn^N}(\wn) & = (I_3\otimes \Pn^{-1}\Pn^N)
  \begin{pmatrix}
   - \epsilon \Dy & \zeron & \zeron
   \\
   \zeron & - \epsilon \Dy & \In
   \\
   \zeron & \zeron & \zeron
  \end{pmatrix}
   \, .
\end{aligned}
\end{equation*}
\label{prop:sat}
\end{proposition}

\begin{remark}
We see from \Cref{prop:ins} and \Cref{prop:sat} that parts of the blocks in the Jacobian of both $J_{\euln}$ and $J_{\satn}$ are obtained directly from the construction of $\euln$. The few remaining parts are obtained by $i)$ matrix multiplications between a diagonal matrix and a non-diagonal one,
 for example $\Un \Dx$ and $ii)$ matrix additions. This leads to few new additional operations and hence efficiency.
\end{remark}
