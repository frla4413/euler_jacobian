\section{Introduction}%
\label{sec:introduction}
Nonlinear systems of partial differential equations are common in computational science and engineering, and present multiple challenges. Stability is needed for reliability and high accuracy for fine solution details. For fast turnaround and timely result delivery, generic systems of nonlinear equations from the discretization of the form
\begin{equation}
  \Fn(\soln) = 0
  \label{eq:intro}
\end{equation}
must be solved efficiently \cite{quarteroni2010numerical}. This is the topic of this paper. Several techniques exist to solve \eqref{eq:intro}, for example dual-time stepping \cite{jameson1991time,nordstrom2019dual}, optimization algorithms \cite{nocedal2006numerical} or iterative methods \cite{quarteroni2010numerical}. Among the classical iterative methods, Newton's method is an effective choice due to its quadratic convergence order. The obvious drawback with Newton's method is that the Jacobian must be known. Methods that bypass this requirement and instead approximate the Jacobian lead to lower convergence orders, a typical example is the Secant method \cite{quarteroni2010numerical}. Alternatively, by using Newton-Krylov methodologies \cite{knoll2004jacobian}, only the action of the Jacobian is required and that can be approximated by $J_{\Fn}(\soln^k) \delta \un \approx (\Fn(\soln^k + \delta \un) - \Fn(\soln^k))/\delta$, where $\delta$ is small and $\un$ depends on the subspaces in the Krylov iterations. The advantage of Newton-Krylov methods is that an explicit Jacobian is never required, but sophisticated preconditioners becomes necessary \cite{brown1990hybrid} instead.

The focus in this paper is to facilitate the use of Newton's method where the key component is an exact explicit form of the Jacobian of \eqref{eq:intro}. To exemplify our technique, we will use finite-difference operators on summation-by-parts (SBP) form \cite{svard2014review} to discretize the incompressible Navier-Stokes (INS) equations in space. The boundary conditions will be weakly imposed via the Simultaneous Approximation Term (SAT) technique \cite{carpenter1994time}. In \cite{nordstrom2019energy}, such a discretization based on the SBP-SAT technique of the nonlinear INS equations was proven to be stable, which is the key prerequisite.

Based on the formulation in \cite{nordstrom2019energy}, we show how the Jacobian can be explicitly calculated. It is also shown that the Jacobian has a block structure, where several blocks are precomputed when forming $\Fn$, making the procedure very efficient. 

To keep the paper focused on the derivation of the Jacobian, we follow \cite{nordstrom2019energy} and consider a Cartesian grid. Exact Jacobians for numerical discretizations have recently been developed in \cite{chan2020explicit} for so-called entropy stable numerical discretizations on SBP form in a periodic setting. Our new technique is not restricted to such specific discretizations, and we include the specific Jacobian related to the boundary conditions. The technique demonstrated in this paper can be used in a straightforward way on any numerical method for IBVPs that can be formulated on matrix-vector form.  In addition, it can be readily extended to curvilinear grids \cite{aalund2019encapsulated}, arbitrary dimensions, and other sets of linear and nonlinear equations. \textcolor{blue}{In principle all that is needed for the existence of the Jacobian is of course that $\Fn$ is differentiable with respect to $\bm{\phi}$. This covers the various nonlinearities that arise in discretizations of the Navier-Stokes equations (both compressible and incompressible). However, the feasibility of explicitly deriving the Jacobian is highly dependent on the way in which $\Fn$ is presented. As we shall see, discretizations on SBP-SAT form are particularly simple to differentiate, making Newton's method an attractive solution method.}

The rest of the paper proceeds as follows. We introduce the continuous problem in \Cref{sec:continuous} and present the semi-discrete formulation in \Cref{sec:semi_discrete}. The Jacobian of the discretization is derived in \Cref{sec:jacobian}. Implicit time integration is discussed in \Cref{sec:fully_discrete} and numerical experiments are performed in \Cref{sec:numerical_experiments}. Finally, conclusions are drawn in \Cref{sec:conclusions}.
